% !TEX encoding = UTF-8
% !TEX TS-program = pdflatex
% !TEX root = ../tesi.tex

%**************************************************************
\chapter{Conclusioni}
\label{cap:conclusioni}
%**************************************************************

\intro{In questo capitolo vengono valutati gli obiettivi raggiunti e infine viene fatta una valutazione personale dello stage svolto.}\\

%**************************************************************
\section{Raggiungimento degli obiettivi}
Nella tabella sottostante vengono rappresentati gli obiettivi raggiunti a fine stage specificando lo stato di completamento di ognuno di essi.\\
\begin{center}
	\begin{table}[h!]
		
		\label{tab:Raggiungimento obiettivi}
		\begin{tabularx}{\textwidth}{|c|p{7cm}|p{2.4cm}|}
			
			\hline
			\textbf{Identificativo} & \centering\textbf{Descrizione} & \textbf{Stato}  \\\hline
			
			O01 & Acquisizione delle competenze sul progetto in generale, il framework Flutter e il linguaggio Dart.  & Completato\\
			\hline
			O02 & Capacità di raggiungere gli obiettivi richiesti in autonomia seguendo il cronoprogramma.  & Completato\\
			\hline	
			O03 & Portare a termine le implementazioni previste con una percentuale di superamento pari all’80\%.  & Completato\\
			\hline
			D01 & Portare a termine le implementazioni previste con una percentuale di superamento pari al 100\%.  & Completato\\
			\hline
			F01 & Realizzazione di una nuova funzionalità per l'app che prevede la gestione Signin con il protcollo Oath2. & Non completato\\
			\hline		
		\end{tabularx}
		\vspace{0.3cm}
		\caption{Raggiungimento obiettivi}
	\end{table}
\end{center}

Si può vedere che in base agli obiettivi stabiliti inizialmente, tutti quelli obbligatori e desiderabili sono stati completati.\\
Per quanto riguarda il requisito facoltativo è stato lasciato da parte seppur fosse molto interessante.\\
Il motivo principale di questa scelta è data dal fatto che la parte di login è stata assegnata all'altro stagista e io non l'ho più vista dettagliatamente.\\
Invece, sempre concordandomi con il tutor Fabio Pallaro, ho optato per concentrarmi su altri aspetti interessanti come la pubblicazione dell'applicazione sul Play Store.


%**************************************************************
\section{Valutazione personale}
Questa esperienza di stage, in sintesi, la posso definire molto piacevole e importante.\\
Mi ha permesso innanzitutto di entrare a far parte di una nuova realtà come quella del lavoro, in un ambiente molto sano e collaborativo.\\
Inoltre sono riuscito a mettere in pratica quanto studiato durante questi tre anni e grazie al lavoro, per lo più autonomo, sono riuscito a individuare i maggiori problemi e a capire come risolverli.\\
Grazie a questa esperienza ho compreso meglio una nuova tecnologia, abbastanza recente, come Flutter che mi ha fatto appassionare alla programmazione di applicazioni mobile nei vari suoi aspetti.\\
In conclusione mi posso ritenere soddisfatto dell'esperienza appena svolta e se in futuro ci saranno altri contatti con l'azienda SyncLab ne sarei profondamente lieto e onorato.


