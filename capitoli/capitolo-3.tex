% !TEX encoding = UTF-8
% !TEX TS-program = pdflatex
% !TEX root = ../tesi.tex

%**************************************************************
\chapter{Framework Flutter e linguaggio Dart}
\label{cap:Framework Flutter e linguaggio Dart}
%**************************************************************

\intro{In questo capitolo viene inizialmente introdotto come un'applicazione mobile può essere sviluppata per poi concentrarsi su Flutter e tutte le sue componenti. Infine saranno mostrate come esempio alcune piccole applicazioni realizzate per imparare a utilizzare il framework Flutter.}\\

%**************************************************************
\section{Sviluppo applicazioni mobile}
Nel quotidiano l'uso dello smartphone è in costante aumento, non solo in Italia ma in tutto il mondo.
Mentre fino a qualche anno fa il cellulare veniva usato solo per telefonare o mandare qualche messaggio, oggi lo smartphone viene utilizzato in qualsiasi ambito: lavoro, comunicare, divertirsi, video, musica o svago.
\subsection{App native}
Android(Java or Kotlin), iOS (Objective C, Swift)
\subsection{Web app}
PolymerJS, Angular e React
\subsection{App Ibride Web View Wrapper}
Ionic, Apache Cordova e PhoneGap
\subsection{App Ibride Compile to Native}
React Native, Ionic, Xamarin e Flutter

\section{Flutter}

\subsection{Dart}

\subsection{Componenti}

\subsubsection{Framework}

\subsubsection{Engine}
Skia
\subsubsection{Embedder}

\subsection{Widget}

\subsubsection{Stateless widget}

\subsubsection{Stateful widget}

\subsubsection{Librerie Material e Cupertino}

\subsubsection{Principali Widget}

\subsection{Rendering e Layout}

\subsection{Esempi piccole applicazioni realizzate}
