% !TEX encoding = UTF-8
% !TEX TS-program = pdflatex
% !TEX root = ../tesi.tex

%**************************************************************
\chapter{Framework Flutter e linguaggio Dart}
\label{cap:Framework Flutter e linguaggio Dart}
%**************************************************************

\intro{In questo capitolo, viene introdotto come un'applicazione mobile può essere sviluppata per poi concentrarsi su Flutter e tutte le sue componenti. Infine saranno mostrate come esempio alcune piccole applicazioni realizzate per imparare a utilizzare il framework Flutter.}\\

%**************************************************************
\section{Sviluppo di applicazioni mobile}
Nel quotidiano, non solo in Italia ma in tutto il mondo, l'uso dello smartphone \cite{statistiche} è in costante aumento.
Mentre fino a qualche anno fa il cellulare veniva usato solo per telefonare o mandare qualche messaggio, oggi lo smartphone viene utilizzato in qualsiasi ambito: lavoro, comunicare, divertirsi, video, musica o svago. \\
Ormai nei cellulari sono presenti applicazioni per qualsiasi esigenza ed è proprio per questo che chi progetta applicazioni ha dovuto considerare lo sviluppo per mobile come fattore primario. Come si vede in Figura 3.1 questo lavoro richiede una moltitudine di passaggi.\\
	\begin{figure}[htbp]	
	\centering
	\includegraphics[width=10cm]{immagini/sviluppoapp.png}
	\caption{Sviluppo applicazioni mobile}
	\label{fig:Sviluppo applicazioni mobile}
\end{figure}

\newpage

Esistono quattro diversi approcci di implementazione \cite{differenza,sviluppo,apptonative}: 
\begin{itemize}
	\item App native; 
	\item Web app; 
	\item App Ibride Web View Wrapper; 
	\item App Ibride Compile to Native.
\end{itemize}

\subsection{App native}
Il metodo nativo \cite{differenza,apptonative} dà la possibilità all'applicazione di integrarsi con la parte hardware del dispositivo, sfruttando così tutte le funzionalità del sistema operativo. \\
Le app native vengono realizzate utilizzando gli strumenti di sviluppo software e la documentazione fornita dai produttori del sistema operativo per il quale si ha l'intenzione di sviluppare.\\
Questo metodo è scelto sopratutto degli sviluppatori attenti alle performance dell'applicazione.\\
I vantaggi principali di sviluppare App native sono:
\begin{itemize}
	\item Maggiore velocità, affidabilità e reattività;
	\item Accesso diretto alla parte hardware e al software installato nel device; 
	\item Notifiche dirette;
	\item Funzionamento offline.
\end{itemize}
Attualmente i sistemi operativi più utilizzati sono \cite{sviluppo}:
\begin{itemize}
	\item Android; 
	\item iOS; 
	\item Windows Phone.\\
\end{itemize}

\subsubsection{Android}
 Android, il cui logo è riportato in Figura 3.2, è il sistema operativo più utilizzato e diffuso. È stato sviluppato da Google ed è stato scelto da multinazionali importanti come Samsung \cite{samsung}, Huawei \cite{huawei} e Amazon \cite{amazon} per il funzionamento dei loro dispositivi.\\
 Il linguaggio per sviluppare un'applicazione Android è Java \cite{java}. Negli ultimi anni è nato anche Kotlin \cite{kotlin}, che è un altro linguaggio ufficiale per la progettazione di applicazioni Android più moderno, meno complesso ma performante e compatibile con l'ambiente Android quanto Java.\\
 \begin{figure}[htbp]	
 	\centering
 	\includegraphics[width=1cm]{immagini/logoandroid.png}
 	\caption{Logo Android}
 	\label{fig:Logo Android}
 \end{figure}

 \newpage

 \subsubsection{iOS}
 iOS, il cui logo è riportato in Figura 3.3, è il sistema operativo sviluppato da Apple per dispositivi iPhone, iPod touch e iPad.\\
 Per un lungo periodo il linguaggio per sviluppare un'applicazione iOS è stato Objective-C che deriva da C e C++. \\
 Per aumentare la produttività, Apple ha lanciato un linguaggio di più alto livello ovvero Swift \cite{swift}. Swift è veloce, leggibile e poco prolisso. Nonostante ciò, Objective-C viene ancora preferito quando si sta lavorando più a basso livello \cite{sviluppo}.\\
\begin{figure}[htbp]	
	\centering
	\includegraphics[width=1cm]{immagini/logoios.jpg}
	\caption{Logo iOS}
	\label{fig:Logo iOS}
\end{figure}

 \subsubsection{Windows Phone}
Windows Phone, il cui logo è riportato in Figura 3.4, è il sistema operativo sviluppato da Microsoft.\\
Il linguaggio utilizzato per sviluppare un'applicazione Windows Phone è C\#, che è un linguaggio semi-compilato orientato agli oggetti.\\
Rispetto a Android e iOS in ambito mobile risulta inferiore. Invece per quanto riguarda sistemi desktop, Windows è una delle migliori.\\
 \begin{figure}[htbp]	
	\centering
	\includegraphics[width=3cm]{immagini/logowindowsphone.png}
	\caption{Logo Windows Phone}
	\label{fig:Logo Windows Phone}
\end{figure}
 
\subsection{Web app}
Una web app \cite{differenza,apptonative} è un'applicazione che funziona come un sito web adattandosi al dispositivo utilizzato.
Queste applicazioni non necessitano di essere installate sugli smartphone e quindi non andranno ad aumentare la memoria utilizzata nel dispositivo.
 Inoltre non possono essere pubblicate sugli Store e quindi non godono di un enorme visibilità.\\
I principali framework e librerie per creare una web app sono:
\begin{itemize}
	\item Angular \cite{angular}; 
	\item PolymerJS \cite{polymer}; 
	\item React \cite{react}.
\end{itemize}
I vantaggi di sviluppare una web app sono:
\begin{itemize}
	\item scrittura con Markup HTML;
	\item non essendo pubblicate sul Market, le web app non devono essere sottoposte al processo di approvazione; 
	\item minor tempo di sviluppo.
\end{itemize}

\newpage

\subsection{App Ibride Web View Wrapper}
Questo tipo di metodo \cite{differenza,apptonative} permette di creare applicazioni senza alcuna conversione del codice in base al sistema operativo.\\
 In pratica l'applicazione rileva inizialmente il sistema operativo utilizzato e successivamente imita l'aspetto dell'interfaccia utente, utilizzando linguaggi di formattazione come CSS \cite{css}, Sass \cite{sass} etc. \\
Le piattaforme più usate sono:
\begin{itemize}
\item Ionic \cite{ionic}; 
\item Apache Cordova \cite{apache}; 
\item PhoneGap \cite{phonegap}.
\end{itemize}
I vantaggi principali di Ionic e delle App Ibride Web View Wrapper sono:
\begin{itemize}
	\item riutilizzo facile del codice; 
	\item Ionic utilizza JavaScript e fornisce un supporto per Angular;
	\item addatamento automatico in base alla piattaforma; 
\end{itemize}
Lo svantaggio principale di questo tipo di applicazioni sta in termini di velocità di esecuzione e questo comporterà prestazioni inferiori rispetto alle applicazioni compilate in nativo.\\

\subsection{App Ibride Compile to Native}
Le applicazioni ibride che compilano in nativo \cite{apptonative} utilizzano un unico linguaggio di programmazione per la scrittura del codice.\\
 Una volta compilato, i componenti dell'interfaccia utente del codice vengono convertiti nei componenti dell'interfaccia utente nativi senza bisogno di configurazioni particolari.\\
Le principali piattaforme utilizzate per compilare in nativo sono:
\begin{itemize}
	\item React Native \cite{reactnative}; 
	\item NativeScript \cite{nativescript}; 
	\item Xamarin \cite{xamarin}; 
	\item Flutter \cite{flutter}.
\end{itemize}
I vantaggi principali di utilizzare piattaforme che compilano in nativo sono:
\begin{itemize}
	\item anche se minore delle App Ibride Web View Wrapper, hanno un elevata riutilizzabilità del codice; 
	\item elevato numero di librerie utilizzabili;  
	\item compilando in nativo offrono prestazioni elevate.
\end{itemize}
\newpage

\section{Flutter}
Flutter \cite{flutterprogramma,flutter,fluttermobile}, il cui logo è riportato in Figura 3.5, è un framework nato abbastanza di recente per lo sviluppo di applicazioni per diverse piattaforme.
È stato ideato da Google come progetto open source e pubblicato ufficialmente per la prima volta a dicembre del 2018 nella versione 1.0 all'evento Flutter Live.
Il 3 marzo 2021 è stata rilasciata la versione 2.0 che consente agli sviluppatori di generare in maniera stabile applicazioni multipiattaforma.
\\
\begin{figure}[htbp]	
	\centering
	\includegraphics[width=3cm]{immagini/flutterlogo.jpg}
	\caption{Logo Flutter}
	\label{fig:Logo Flutter}
\end{figure}
\\
Flutter offre una vasta serie di librerie di elementi di interfaccia utente e combina la facilità di sviluppo con prestazioni simili alle prestazioni native, mantenendo una corretta distinzione visiva tra le diverse piattaforme senza che il programmatore debba prestare particolari attenzioni.
Viene utilizzato sopratutto per applicazioni Android e iOS e funziona come una vera applicazione nativa.\\
Utilizzando lo stesso codebase, ovvero l'intera collezione di codice sorgente usata per costruire l'applicazione, è possibile creare l'applicazione per diverse piattaforme.\\
Il linguaggio di programmazione di Flutter è Dart, sviluppato anche esso da Google, ed è stato pensato per sostituire JavaScript.\\
Flutter è completamente gratuito e come possiamo vedere dalla figura 3.6, già nell'ultimo anno la sua popolarità e il suo utilizzo è cresciuto notevolmente.\\

\begin{figure}[htbp]	
	\centering
	\includegraphics[width=12cm]{immagini/statisticheflutter.png}
	\caption{Statistiche dei programmi usati per sviluppare app mobile nel 2019 e 2020 \cite{stat}}
	\label{fig:Statistiche dei programmi usati per sviluppare app mobile nel 2019 e 2020} 
\end{figure}
\newpage
Anche se molto recente, nel Google Play Store possiamo contare oltre 50000 applicazioni Flutter.\\
I vantaggi di utilizzare Flutter sono:
\begin{itemize}
	\item un codebase per tutte le piattaforme; 
	\item utilizzo di Dart \hyperref[sec:Dart]{(vedi sezione 3.2.1)} che è un linguaggio facile da apprendere;  
	\item è più facile sviluppare applicazioni con Flutter che quindi entrano sul mercato più in fretta;
	\item si basa sul principio \textit{Tutto è un Widget} che sarà spiegato nella sezione Widget \hyperref[sec:Widget]{(vedi sezione 3.2.4)};
	\item basso consumo di risorse;
	\item esecuzione performante delle app native su smartphone;
	\item ottima interfaccia utente che può essere anche personalizzata;
	\item l’hot reload permette di vedere le modifiche in tempo reale accelerando lo sviluppo.\\
\end{itemize}
Invece tra gli svantaggi possiamo citare che:
\begin{itemize}
	\item Dart è un linguaggio nuovo e quindi non è molto diffuso; 
	\item esendo nuovo, mancano librerie di terze parti che facilitano lo sviluppo;  
	\item vengono create app di grandi dimensioni rispetto agli altri framework o anche a Java stesso.\\
\end{itemize}
\subsection{Dart}
\label{sec:Dart}
Dart \cite{dart,dartstoria} è un linguaggio di programmazione sviluppato da Google e presentato per la prima volta il 10 ottobre del 2011 alla conferenza 'GOTO Aarhus 2011'.\\
Lo scopo principale è quello di sostituire JavaScript per lo sviluppo delle applicazioni.\\
Dart costituisce la base di Flutter e inoltre supporta molte attività di sviluppo di base come la formattazione, l'analisi, il test del codice...\\
Dart è type-safe \cite{dartover}. Inoltre i valori in Dart non possono essere \textit{null} tranne nei casi in cui viene indicato che questi valori possono esserlo, così da evitare possibili errori nel codice.\\
Dart ha un vasto numero di librerie di base e di pacchetti per le API aggiuntive.\\
Permette di scrivere programmi attraverso due distinte piattaforme: 
\begin{itemize}
	\item Dart Native: per le applicazioni sviluppate per dispositivi mobili e desktop. Include sia una macchina virtuale Dart con compilazione JIT (just-in-time), ovvero il codice viene compilato durante l'esecuzione del programma e facilita l'hot reload, sia un compilatore AOT (Ahead-of-Time) per la produzione di codice macchina, ovvero  viene compilato prima dell'esecuzione, tipicamente durante l'installazione del programma, in modo da migliorare le prestazioni diminuendo i tempi di avvio e evitando la fase di compilazione durante l'esecuzione del programma;  
	\item Dart Web: per le applicazioni destinate al Web. Include sia un compilatore del tempo di sviluppo (\textit{dartdevc}) che consente di eseguire il debug dell'applicazione nel browser Chrome e vedere le modifiche quasi immediatamente, che un compilatore del tempo di produzione (\textit{dart2js}) che fornisce suggerimenti per migliorare il codice Dart e rimuovere il codice inutilizzato. Entrambi i compilatori traducono Dart in JavaScript.\\
\end{itemize}

\subsection{Componenti}
Flutter, come si vede nella Figura 3.7, è formato da un'architettura a strati \cite{flutterd, flutterdettagli} e i suoi tre strati fondamentali sono:
\begin{itemize}
	\item Framework; 
	\item Engine;  
	\item Embedder.\\
\end{itemize}
\begin{figure}[htbp]	
	\centering
	\includegraphics[width=13cm]{immagini/composizione.png}
	\caption{Architettura a strati di Flutter}
	\label{fig:Architettura a strati di Flutter}
	\cite{flutterdettagli}
\end{figure}

\newpage

\subsubsection{Embedder}
Embedder è il livello più basso dell'architettura.\\
Il linguaggio utilizzato è definito in base alla piattaforma e attualmente può essere: Java e C++ per Android, Objective-C/Objective-C++ per iOS e macOS e C++ per Windows e Linux.\\
Ha lo scopo di legare il rendering della schermata nativa, la gestione degli eventi e altri elementi.\\
Per fare ciò lo strato Embedder interagisce con lo strato Engine tramite delle API C/C++.\\
Inoltre lo strato Embedder è composta da una Shell che ospita anche la Dart VM.		\\
Ogni Shell è specifica per ogni piattaforma e offre un accesso alle API native della piattaforma in questione.

\subsubsection{Engine}
Engine è lo strato intermedio dell'architettura.\\
Il linguaggio utilizzato è principalmente il C++ e il C per rendere più veloci ed efficienti le applicazioni realizzate in Flutter.\\
Contiene componenti di basso livello essenziali per il funzionamento del framework.\\
All'interno troviamo il motore grafico \textit{Skia}, una libreria grafica 2D open source scritta in C++ creata da Google, e le shell a cui è possibile accedervi tramite le API esposte dalla libreria \textit{dart:ui}.

\subsubsection{Framework}
È lo strato più importantedell'architettura.\\
Il linguaggio utilizzato è Dart \hyperref[sec:Dart]{(vedi sezione 3.2.1)}.\\
All'interno sono presenti classi fondamentali di base e servizi di base come le animazioni.\\
Inoltre è presente il Rendering che permette di gestire il layout attraverso un albero di oggetti che viene visualizzato
a schermo e che si aggiorna automaticamente.\\
Sempre all'interno di questo strato sono presenti i Widget e le due librerie: Material e Cupertino che sono approfonditi nella successiva sezione 3.2.3.

\newpage

\subsection{Librerie Material e Cupertino}
Come si vede in Figura 3.8, le due principali librerie di Flutter sono Material e Cupertino \cite{flutterd}. \\ 
\begin{figure}[htbp]	
	\centering
	\includegraphics[width=14cm]{immagini/librerieCM.jpg}
	\caption{Librerie Material e Cupertino \cite{lib}}
	\label{fig:Librerie Material e Cupertino}
\end{figure}
\\
All'interno di Flutter è possibile capire su che piattaforma si sta eseguendo l'applicazione attraverso \textit{Platform.isIOS}.
 Se ritornerà un valore booleano uguale a \textit{true} significa che saremo su iOS altrimenti saremo su Android.\\
La libreria Cupertino permette di implementare un design specifico per le applicazioni iOS.
A differenza della Cupertino la libreria Material permette di implementare un design specifico per le applicazioni Android.

\subsection{Widget}
\label{sec:Widget}
Flutter è basato sul principio \textit{Tutto è un widget} \cite{flutterprogramma} in quanto l'interfaccia di un programma è composta da diversi widget nidificati.\\
Ogni widget può essere del testo o un pulsante o un qualsiasi altro elemento grafico che contiene varie caratteristiche.\\
Tutti questi widget possono influenzare altri widget nella costruzione dell'applicazione.\\
Il vantaggio principale di questa struttura a widget consiste nella flessibilità, invece lo svantaggio sta nel fatto che tutti i widget sono situati nel codice sorgente del programma e pertanto risulteranno fortemente nidificati.
\newpage
Come si vede in Figura 3.9, il framework contiene due classi principali di widget \cite{ flutterd, state}: 
\begin{itemize}
	\item Stateless widget;   
	\item Stateful widget.
\end{itemize}

\begin{figure}[htbp]	
	\centering
	\includegraphics[width=12cm]{immagini/state.jpeg}
	\caption{Stateless widget e Stateful widget \cite{state}}
	\label{fig:Stateless widget e Stateful widget}
\end{figure}
\subsubsection{Stateless widget}
Gli stateless widget non hanno uno stato mutabile e quindi non cambieranno nel tempo, neanche in seguito a comportamenti effettuati dall'utente.\\
Alcuni esempi di questi widget sono: Text, Row, Column e Container.\\
Per creare uno stateless widget bisogna estendere la classe \textit{StatelessWidget} che richiede l'override del metodo \textit{build()}.
Questo metodo viene invocato la prima volta per costruire l'albero dei widget e poi ogni volta che le loro dipendenze cambiano.\\
Si può usare uno stateless widget solamente quando i campi dei widget non cambiano nel tempo neanche dopo azioni dell'utente.\\
Negli altri casi sarà da utilizzare uno stateful widget.

\subsubsection{Stateful widget}
Gli stateful widget sono dinamici e mutano nel tempo in base all'interazione dell'utente o in base ad altri fattori.\\
Alcuni esempi di questi widget sono: Image, Form e Checkbox.\\
Per creare uno stateful widget bisogna estendere la classe \textit{SteatefulWidget}.\\
Come si intuisce dal nome, questi widget dipendono dallo stato dell'oggetto. Infatti ogni volta che si vuole modificare lo stato di un qualsiasi oggetto bisogna chiamare il metodo \textit{setState()} segnalando così al framework di aggiornare l'interfaccia utente chiamando il metodo \textit{build()} e tenendo in considerazione i nuovi stati degli oggetti assegnati nel metodo appena descritto. \\
Dobbiamo quindi creare uno stateful widget quando il widget potrebbe cambiare durante il suo ciclo di vita. 

\newpage

\subsubsection{Principali Widget}
Di seguito saranno spiegati brevemente i widget più comuni e utilizzati in Flutter.

\paragraph{Scaffold}
Il widget Scaffold \cite{scaffold} (vedi Figura 3.10)implementa la struttura del layout visivo.\\
Contiene principalmente 5 elementi:
\begin{itemize}
	\item \textbf{appBar} che viene descritta in seguito;   
	\item \textbf{body} che rappresenta il corpo situato sotto l'Appbar;
	\item \textbf{floatingActionButton} che è un bottone che è situato di default in basso a destra;   
	\item \textbf{drawer} che è un menù laterale visibile dall'utente scorrendo da sinistra a destra o viceversa. All'interno di questo menù sono presenti altri widget che permettono di effettuare diverse azioni;
	\item \textbf{bottomNavigationBar} che permette di gestire un menù nella parte inferiore dell'applicazione.
\end{itemize}

\begin{figure}[htbp]	
	\centering
	\includegraphics[width=8cm]{immagini/scaffold.png}
	\caption{Scaffold widget}
	\label{fig:Scaffold widget}
\end{figure}

\newpage

\paragraph{AppBar}
L'AppBar viene gestita da Scaffold ed è solitamente situata nella parte superiore dell'applicazione come si vede in Figura 3.11. Espone solitamente altri widget che permettono di eseguire una o più azioni.
AppBar ha diversi elementi grafici come elevazione e titolo.
\begin{figure}[htbp]	
	\centering
	\includegraphics[width=8cm]{immagini/appbar.png}
	\caption{Appbar widget}
	\label{fig:Appbar widget}
\end{figure}

\paragraph{Row, Column} (vedi Figura 3.12)
Il widget Row permette di creare una famiglia di widget in una riga orizzontale invece il widget Column permette di creare sempre una famiglia di widget ma in verticale.\\
Entrambi non sono scorrevoli quindi nel caso si abbia bisogno di una famiglia di widget scorrevoli bisogna utilizzare il widget \textit{ListView}.
\begin{figure}[htbp]	
	\centering
	\includegraphics[width=10cm]{immagini/rowcolumn.png}
	\caption{Row e Column widget}
	\label{fig:Row e Column widget}
\end{figure}

\newpage

\paragraph{Container} (vedi Figura 3.13)
Il widget Container funge da contenitore all'interno dell'applicazione e può racchiudere altri widget.\\
Ha un margine, un bordo e un padding di default che possono essere modificati.\\
Oltre a questi elementi, questo widget contiene altri elementi che permettono di decorare la sua area di competenza.
\begin{figure}[htbp]	
	\centering
	\includegraphics[width=8cm]{immagini/container.png}
	\caption{Container widget}
	\label{fig:Container widget}
\end{figure}

\paragraph{Text} (vedi Figura 3.14)
Il widget Text consente di creare una porzione di testo all'interno dell'applicazione che può essere decorata in base alle diverse esigenze.
\begin{figure}[htbp]	
	\centering
	\includegraphics[width=5cm]{immagini/text.png}
	\caption{Text widget}
	\label{fig:Text widget}
\end{figure}

\paragraph{Image}
Consente di impostare all'interno dell'applicazione un'immagine.\\
Ci sono vari formati disponibili ovvero JPEG, PNG, GIF.\\
I modi per ottenere l'immagine all'interno dell'applicazione sono:
\begin{itemize}
	\item \textbf{new Image} che permette di ottenere un'immagine da un  ImageProvider (utilizza imageCache globale per memorizzare nella cache le immagini);   
	\item \textbf{new Image.asset} che permette di ottenere un'immagine da un AssetBundle (una raccolta di risorse utilizzate dall'applicazione) utilizzando una chiave;
	\item \textbf{new Image.network} che permette di ottenere un'immagine da un URL;   
	\item \textbf{new Image.file} che permette di ottenere un'immagine da un File;
	\item \textbf{new Image.memory} che permette di ottenere un'immagine da un Uint8List.
\end{itemize}

\paragraph{Icon}
Il widget Icon permette di inserire icone all'interno dell'applicazione.\\
Come si vede in Figura 3.15, le icone sono quadrate e non sono interattive.
\begin{figure}[htbp]	
	\centering
	\includegraphics[width=5cm]{immagini/icon.png}
	\caption{Alcune icone di Flutter}
	\label{fig:Alcune icone di Flutter}
\end{figure}

\paragraph{ElevatedButton} (vedi Figura 3.16)
Il widget ElevatedButton permette di inserire un bottone con uno stile di default all'interno dell'applicazione.\\
Contiene l'elemento \textit{onPressed} che permette di eseguire operazioni quando il bottone viene premuto. Nel caso onPressed venga impostato uguale a \textit{null} il bottone sarà disabilitato.
\begin{figure}[htbp]	
	\centering
	\includegraphics[width=8cm]{immagini/button.png}
	\caption{ElevatedButton widget}
	\label{fig:ElevatedButton widget}
\end{figure}

\newpage

\subsection{Esempi piccole applicazioni realizzate}
Di seguito verranno mostrate alcune applicazioni realizzate seguendo un corso su Udemy \cite{corso} per imparare a usare il framework Flutter.

\subsubsection{Applicazione base}
Dopo aver effettuato la giusta configurazione e scaricato tutto il necessario, per creare la prima applicazione di base messa a disposizione da Flutter basterà digitare sul terminale il comando: \textit{flutter create "nome\_applicazione"}.\\
Questo ci permette di creare la nostra prima applicazione base come esposto in Figura 3.17.\\

\begin{figure}[htbp]	
	\centering
	\includegraphics[width=6cm]{immagini/base.jpeg}
	\caption{Applicazione base}
	\label{fig:Applicazione base}
\end{figure}

\newpage

\subsubsection{Applicazione 1}
Questa applicazione (vedi Figura 3.18) permette di rispondere a una serie di domande e successivamente, a quiz terminato, ci sarà una valutazione finale in base alle risposte assegnate.\\
Una volta terminato, il quiz può essere ritentato.\\
In particolare la creazione di questa applicazione mi ha permesso di apprendere meglio il funzionamento di alcuni widget fondamentali.\\
Tra questi sono stati visti principalmente i seguenti: AppBar, Text, Container, ElevatedButton.\\
Inoltre mi ha permesso di capire gli attributi dei widget precedentemente elencati e mi ha fatto comprendere il funzionamento dello \textit{setState()}.\\

\begin{figure}[htbp]	
	\centering
	\includegraphics[width=6cm]{immagini/app1.jpeg}
	\caption{Applicazione 1}
	\label{fig:Applicazione 1}
\end{figure}

\newpage

\subsubsection{Applicazione 2}
Questa applicazione (vedi Figura 3.19) permette di inserire delle transazioni che un utente ha effettuato così da monitorare le spese sostenute.\\
In particolare mi ha permesso di comprendere alcuni attributi e widget più complessi rispetto a quelli base, studiare il layout in modo che si adattasse alle dimensioni del cellulare, configurare una pagina in modo scrollabile e gestire widget a comparsa.\\

\begin{figure}[htbp]	
	\centering
	\includegraphics[width=6cm]{immagini/app2.jpeg}
	\caption{Applicazione 2}
	\label{fig:Applicazione 2}
\end{figure}

\newpage

\subsubsection{Applicazione 3}
Quest'ultima applicazione (vedi Figura 3.20) permette di gestire diverse categorie di cibo, i quali contengono diverse ricette che possono essere aggiunte ai preferiti.\\
In particolare mi ha permesso di capire il corretto funzionamento delle Card \cite{card} (una porzione ben definita di testo che può contenere vari elementi come titolo, sottotitolo, immagini, etc.), delle liste e come gestire nel menu laterale (drawer) i preferiti e i filtri.\\

\begin{figure}[htbp]	
	\centering
	\includegraphics[width=6cm]{immagini/app3.jpeg}
	\caption{Applicazione 3}
	\label{fig:Applicazione 3}
\end{figure}






