% !TEX encoding = UTF-8
% !TEX TS-program = pdflatex
% !TEX root = ../tesi.tex

%**************************************************************
\chapter{Sportwill}
\label{cap:Sportwill}
%**************************************************************

\intro{In questo capitolo viene descritto il progetto, le tecnologie e gli strumenti utilizzati. Viene inoltre effettuata un'analisi dei requisiti per poi definire come le varie funzionalità sono state implementate. }\\

%**************************************************************

\section{Descrizione progetto}
Sportwill è un'applicazione che permette di gestire eventi sportivi.\\
Dopo essersi autenticati al sistema si possono visualizzare tutte le attività presenti, passate e future attraverso delle card.\\
 È possibile cercare queste attività utilizzando vari filtri presenti oppure scorrendo la pagina.\\ 
 Ogni utente può gestire i propri eventi creando nuove attività e modificandole successivamente nell'apposita sezione.\\
 Una volta che un utente vuole far partire un nuovo evento dovrà andare sulle specifiche dell'attività e premere sull'apposito bottone per farla iniziare e successivamente la medesima potrà essere messa anche in pausa.\\
 Una volta iniziata l'attività, verranno presi i dati relativi alla posizione così da poter far vedere agli altri utenti in una mappa la posizione in tempo reale.\\
 Grazie a questa mappa gli altri utenti premendo sull'apposito bottone segnaleranno a chi ha fatto partire l'evento che vogliono partecipare anche loro così da poterlo raggiungere.
 
 \newpage
 
 Di seguito vengono riportate due immagini dell'applicazione.\\
 La prima rappresenta la pagina principale dell'applicazione alla quale si arriva dopo essersi autenticati al sistema mentre la seconda rappresenta i dettagli di una nuova attività alla quale posso partecipare premendo sul bottone \textit{Partecipo anche io!}.\\
 
\begin{figure}[htbp]
	\begin{minipage}[b]{0.47\textwidth}
		\centering
		\includegraphics[width=6cm]{immagini/colori.jpeg}
		\caption{Pagina principale}
		\label{fig:Pagina principale}
	\end{minipage}
	\hfill
	\begin{minipage}[b]{0.47\textwidth}
		\centering
		\includegraphics[width=6cm]{immagini/partecipo.jpeg}
		\caption{Schermata di partecipazione}
		\label{fig:Schermata di partecipazione}
	\end{minipage}
\end{figure}
 
 \newpage

\subsection{Organizzazione del progetto}

All'interno del progetto i file sono stati suddivisi nel varie sottocartelle in base al loro compito.\\
Oltre al file main.dart, che è il file principale dell'applicazione, sono presenti altre 4 cartelle che permettono di organizzare meglio il lavoro:
\begin{itemize}
	\item \textbf{models}: che permette di gestire le eccezioni;
	\item \textbf{providers}: che permette di gestire il modello dei dati e fare le varie chiamate al backend;
	\item \textbf{screens}: che crea tutte le pagine dell'applicazione con gli appositi widget;
	\item \textbf{widgets}: che offrono supporto agli screens tramite la creazione di widget.
\end{itemize}
\begin{figure}[htbp]	
	\centering
	\includegraphics[width=5cm]{immagini/struttura.png}
	\caption{Struttura progetto}
	\label{fig:Struttura progetto}
\end{figure}

\newpage

\section{Tecnologie e strumenti}

\subsection{Android Studio}

\begin{figure}[htbp]	
	\centering
	\includegraphics[width=4cm]{immagini/logoandroidstudio.png}
	\caption{Logo Android Studio}
	\label{fig:Logo Android Studio}
\end{figure}

\subsection{GitLab}

\begin{figure}[htbp]	
	\centering
	\includegraphics[width=4cm]{immagini/logogitlab.png}
	\caption{Logo GitLab}
	\label{fig:Logo GitLab}
\end{figure}

\subsection{Database}

\begin{figure}[htbp]	
	\centering
	\includegraphics[width=4cm]{immagini/logodbeaver.png}
	\caption{Logo DBeaver}
	\label{fig:Logo DBeaver}
\end{figure}

\subsection{Backend}
Spring boot e Spring jpa

\begin{figure}[htbp]	
	\centering
	\includegraphics[width=4cm]{immagini/springlogo.png}
	\caption{Logo Spring}
	\label{fig:Logo Spring}
\end{figure}

\newpage

\section{Analisi dei requisiti}
Tipologie di utenti: utente autenticato(come ospite o normale), utente non autenticato.

\subsection{Casi d'uso}

\subsubsection{ UC1 - Visualizzazione mappa}
\begin{itemize}
	\item\textbf{Attori Primari:} .
	\item\textbf{Descrizione:} .
	\item\textbf{Scenario principale:} .
	\item\textbf{Precondizione:} .
	\item\textbf{Postcondizione:} .
\end{itemize}

\subsubsection{ UC2 - Visualizzazione mappa schermo intero}
\begin{itemize}
	\item\textbf{Attori Primari:} .
	\item\textbf{Descrizione:} .
	\item\textbf{Scenario principale:} .
	\item\textbf{Precondizione:} .
	\item\textbf{Postcondizione:} .
\end{itemize}

\subsubsection{ UC3 - Filtro base}
\begin{itemize}
	\item\textbf{Attori Primari:} .
	\item\textbf{Descrizione:} .
	\item\textbf{Scenario principale:} .
	\item\textbf{Precondizione:} .
	\item\textbf{Postcondizione:} .
\end{itemize}

\subsubsection{ UC4 - Filtro avanzato}
\begin{itemize}
	\item\textbf{Attori Primari:} .
	\item\textbf{Descrizione:} .
	\item\textbf{Scenario principale:} .
	\item\textbf{Precondizione:} .
	\item\textbf{Postcondizione:} .
\end{itemize}

\newpage

\subsection{Tracciamento requisti}
requisiti funzionali e di vincolo

tabelle

\newpage

\section{Implementazioni}
Oltre al completamento dei requisiti sono state effettuate altre implementazioni.\\
Nel complesso le funzionalità realizzate durante lo stage sono state: 
\begin{itemize}
	\item Logo;
	\item Filtro di ricerca testo;
	\item Pagina Modifica e campi obbligatori;
	\item Mappa percorso;
	\item Aggiornamento automatico mappa;
	\item Mappa schermo intero;
	\item Colori;
	\item Pagina Modifica;
	\item Eliminazione, Modifica e Aggiunta di un'attività;
	\item Filtro avanzato di ricerca attività.
\end{itemize}
Tutti i servizi sono stati rilasciati su un'apposita repository aziendale su \textit{GitLab}.

\subsection{Logo}
È stato modificato il logo di base di Flutter con quello di Sportwill.\\
Per  fare ciò è stato inserita l'immagine del logo all'interno della seguente cartella \textit{app/src/main/res/mipmap/}.\\
In alternativa si poteva definire nel file \textit{pubspec.yaml} il percorso del logo dell'applicazione e poi inserire l'immagine del logo nella cartella appena definita.\\

\begin{figure}[htbp]	
	\centering
	\includegraphics[width=6cm]{immagini/logosportwill.png}
	\caption{Logo Sportwill}
	\label{fig:Logo Sportwill}
\end{figure}

\newpage

\subsection{Filtro di ricerca testo}
All'interno della pagina principale dell'applicazione ovvero la pagina \textit{uscite\_overview\_screen.dart} è stato aggiunto un widget che permette di ricercare le varie card in base all'evento, al luogo o alla persona.\\
Digitando nella barra di ricerca quindi si potranno ricercare solo le card a cui si è interessati.\\
Per fare ciò è stato inserita una funzione chiamata \textit{ricerca()} che ritorna una TextField come Widget nel file \textit{uscite\_overview\_screen.dart}.\\

\begin{figure}[htbp]	
	\centering
	\includegraphics[width=6cm]{immagini/filtrotesto.jpeg}
	\caption{Filtro di ricerca testo}
	\label{fig:Filtro di ricerca testo}
\end{figure}

\newpage

\subsection{Campi obbligatori e pagina modifica e crea}
All'interno di questa pagina sono stati tolti tutti i campi obbligatori non necessari.\\
Per fare ciò sono stati tolti tutti i validator non necessari nel file \textit{pianifica\_screen.dart} e lasciato gli altri validator con le apposite funzioni.\\
Inoltre all'interno di questa pagina sono stati aggiustati i campi degli orari che nella funzione \textit{selezionaOra()} tornava sempre i minuti attuali.\\
Per rendere all'utente più comprensibile il salvataggio è stato sostituita l'icona di salvataggio con l'icona \textit{Icon(Icons.save)}.\\

\begin{figure}[htbp]	
	\centering
	\includegraphics[width=6cm]{immagini/modifica.jpeg}
	\caption{Campi obbligatori}
	\label{fig:Campi obbligatori}
\end{figure}

\newpage

\subsection{Colori}
All'interno dei vari file sono stati cambiati diversi colori in modo che venissero utilizzati principalmente quelli dell'applicazione, ovvero il bianco e l'arancione.\\

\begin{figure}[htbp]	
	\centering
	\includegraphics[width=6cm]{immagini/colori.jpeg}
	\caption{Colori}
	\label{fig:Colori}
\end{figure}


\subsection{Eliminazione, Modifica e Aggiunta di un'attività}
In seguito alle modifiche di vari campi sono state sistemate le pagine di eliminazione, modifica e aggiunta di un'attività con nuovi valori e formati in modo che venissero rispettate le richieste del backend.\\
In particolar modo sono stati sistemati i valori che riguardavano la data e l'orario con i nuovi formati richiesti e la lunghezza e il numero di partecipanti che richiedevano un double e un intero.

\newpage

\subsection{Mappa percorso}
Per gestire la mappa sono stati creati due providers:
\begin{itemize}
	\item location.dart: che è il modello e contiene tutti i dati che riguardano una posizione;
	\item locations.dart: che permette di chiamare il backend per ottenere i vari valori da visualizzare nella mappa.
\end{itemize}
La mappa viene creata nel file \textit{dettaglio\_screen.dart} e viene mostrata solo se c'è almeno una posizione per l'attività selezionata.\\
Per visualizzare la mappa viene usato Leaflet e viene creato un array con tutte le posizioni che vengono poi disegnate sulla mappa mettendo i marker alla prima posizione e all'ultima in ordine temporale.\\

\begin{figure}[htbp]	
	\centering
	\includegraphics[width=6cm]{immagini/mappa.jpeg}
	\caption{Mappa percorso}
	\label{fig:Mappa percorso}
\end{figure}

\newpage

\subsection{Aggiornamento automatico mappa}
Per fare in modo che la mappa si aggiornasse in modo automatico è stato inserito un Timer che ogni 30 secondi va a chiamare la funzione che ritorna tutte le posizioni di quell'attività.\\
Per evitare che il Timer venisse ricostruito ogni volta è stato usato un valore booleano che permette la costruzione del Timer solo la prima volta.\\

\begin{figure}[htbp]	
	\centering
	\includegraphics[width=14cm]{immagini/automatico.png}
	\caption{Aggiornamento automatico mappa}
	\label{fig:Aggiornamento automatico mappa}
\end{figure}

\newpage

\subsection{Mappa schermo intero}
Per fare in modo che la mappa fosse più utilizzabile dall'utente è stata creata la possibilità di vederla a schermo intero.
Per fare ciò è stato usato un valore booleano, ovvero \textit{ingradisci} inizialmente uguale a false, che se settato a true permetteva di vedere nello schermo solo la mappa.\\
Cliccando sul bottone \textit{schermo intero} veniva settato questo valore a true e veniva ricostruita la pagina invece premendo sull'icona con la croce veniva settato il valore a false ricostruendo sempre la pagina.\\

\begin{figure}[htbp]	
	\centering
	\includegraphics[width=6cm]{immagini/mappaintero.jpeg}
	\caption{Mappa schermo intero}
	\label{fig:Mappa schermo intero}
\end{figure}

\newpage

\subsection{Filtro avanzato di ricerca attività}
Per creare il filtro avanzato è stata creata una funzione \textit{showFilterMenu()} nel file \textit{uscite\_overview\_screen.dart} che contiene una \textit{showDialog}.\\
Per aprire il menù dei filtri bisognerà premere sull'icona dei filtri situata in basso a destra.\\
All'interno di questo menu è possibile filtrare le attività in base allo sport, alla data e se si tratta di tue attività.\\
Per applicare i vari filtri, ovvero quando si preme sull'icona check, è stata utilizzato un booleano che ricostruisce la pagina  con filtro uguale a true e va a chiamare nel provider \textit{uscite.dart} la funzione \textit{findBySporteData} in base ai valori messi nel filtro.\\
L'icona con la croce, una volta premuta, setta il valore del filtro a false e ricostruisce la pagina chiudendo la showdialog dei filtri e cancellando i filtri precedentemente applicati.\\
Per poter capire se abbiamo applicato i filtri basterà guardare se l'icona dei filtri ha il colore verde.\\
Invece se non avremo filtri applicati l'icona avrà il solito colore bianco.\\

\begin{figure}[htbp]	
	\centering
	\includegraphics[width=6cm]{immagini/filtroavanzato.jpeg}
	\caption{Filtro avanzato di ricerca attività}
	\label{fig:Filtro avanzato di ricerca attività}
\end{figure}

\newpage

\subsection{Pubblicazione applicazione Play Store}


