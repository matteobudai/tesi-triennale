% !TEX encoding = UTF-8
% !TEX TS-program = pdflatex
% !TEX root = ../tesi.tex

%**************************************************************
\chapter{Introduzione}
\label{cap:introduzione}
%**************************************************************

\intro{In questo capitolo viene descritta l'azienda che ha ospitato lo stage, le metodologie utilizzate e come viene organizzato il lavoro.}\\

%\noindent Esempio di utilizzo di un termine nel glossario \\
%\gls{api}. \\

%\noindent Esempio di citazione in linea \\
%\cite{site:agile-manifesto}. \\

%\noindent Esempio di citazione nel pie' di pagina \\
%citazione\footcite{womak:lean-thinking} \\

%**************************************************************
\section{L'azienda}

Sync Lab \cite{synclab}, il cui logo è riportato in Figura 1.1, nasce nel 2002 come Software house e si è trasformata rapidamente in System Integrator attraverso uno studiato processo di maturazione delle competenze tecnologiche, metodologiche ed applicative nel dominio del software.
\\
\begin{figure}[htbp]	
		\centering
		\includegraphics[width=7cm]{immagini/logo.png}
		\caption{Logo aziendale Sync Lab}
		\label{fig:Logo aziendale Sync Lab}
\end{figure}
\\
\\
In seguito all'apertura della sede principale di Napoli, Sync Lab è cresciuta esponenzialmente nel mercato ICT (Information and Communication Technologies) e ha consolidato ottimi rapporti con clienti e partner.
Attualmente l'aziende ha più di 150 clienti diretti e finali e vanta un organico di oltre 200 dipendenti, una solida base finanziaria e come si può vedere in Figura 1.2, ha un'ottima diffusione nel territorio italiano attraverso le sue cinque sedi: Napoli, Roma, Milano, Padova e Verona.
\\
\begin{figure}[htbp]	
	\centering
	\includegraphics[width=8cm]{immagini/sedi.png}
	\caption{Sedi Sync Lab}
	\label{fig:Sedi Sync Lab}
\end{figure}
\\
Sync Lab, propone sul mercato interessanti e innovativi prodotti software, nati nel proprio laboratorio di ricerca e sviluppo. Attraverso questi prodotti, ha gradualmente conquistato significativamente fette di mercato nei seguenti settori: mobile, videosorveglianza e sicurezza delle infrastrutture informatiche aziendali.

%**************************************************************
\section{Metodologie utilizzate e principali prodotti}
L’azienda adotta un modello di sviluppo agile che pone le proprie basi nel metodo Scrum \cite{scrum}. Gli stakeholders, infatti, vengono costantemente coinvolti nel processo di
sviluppo del prodotto per raccogliere feedback. Gli obiettivi si possono riassumere in tre punti fondamentali:
\begin{itemize}
	\item Comprendere attentamente il contesto operativo del cliente; 
	\item Fornire al cliente un supporto mirato; 
	\item Accelerare e favorire la formazione di soluzioni. 
\end{itemize}
In base a questi principi Sync Lab raggiunge i propri obiettivi grazie a:
\begin{itemize}
	\item Consulenza; 
	\item Fornitura; 
	\item Sviluppo; 
	\item Manutenzione. 
\end{itemize}

Nell'ambito di prodotti e innovazioni, l'azienda ne può vantare un buon numero.
Tra questi troviamo: 
\begin{itemize}
	\item \textbf{SynClinic} (vedi Figura 1.3) che è un  software integrato per la gestione delle strutture sanitarie, che permette di gestire, organizzare e monitorare tutte le fasi del percorso di cura del paziente; \\
	\begin{figure}[htbp]	
		\centering
		\includegraphics[width=7cm]{immagini/synClinic.jpg}
		\caption{SynClinic}
		\label{fig:SynClinic}
	\end{figure}
	\\
	\item \textbf{DPS 4.0} che permette di gestire la General Data Protection Regulation (GDPR) Privacy in pochi semplici passi con una soluzione guidata per aggiornare e modificare i documenti di privacy in modo conforme agli standard di riferimento; \\
	\item \textbf{StreamLog} che permette di gestire la compliance al provvedimento del Garante per la protezione dei dati personali relativo agli Amministratori di Sistema (AdS). In particolare, permette di soddisfare i requisiti fissati dal Garante; \\
	\item \textbf{StreamCrusher} che è una tecnologia che aiuta ad essere bene informati su quando bisogna prendere decisioni di business, ad identificare velocemente criticità ed a riorganizzare i processi in base a nuove esigenze; \\
	\item \textbf{Wave} che si propone come integrazione tra i mondi della Videosorveglianza e quello dei Sistemi Informativi Territoriali (GIS) abilitando il controllo totale dell'area da sorvegliare; \\
	\item \textbf{Seastream} che mette a disposizione un sistema di monitoraggio avanzato delle flotte armatoriali operative in tutto il mondo e una piattaforma integrata di servizi per gli operatori in ambito portuale. \\
\end{itemize}



%**************************************************************

