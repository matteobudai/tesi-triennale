% !TEX encoding = UTF-8
% !TEX TS-program = pdflatex
% !TEX root = ../tesi.tex

%**************************************************************
% Sommario
%**************************************************************
\cleardoublepage
\phantomsection
\pdfbookmark{Sommario}{Sommario}
\begingroup
\let\clearpage\relax
\let\cleardoublepage\relax
\let\cleardoublepage\relax

\chapter*{Sommario}

Il presente documento descrive lo stage da me svolto nel periodo che va dal 28/06/2021 al 20/08/2021, della durata di trecentoventi ore, presso l'azienda Sync Lab s.r.l. nella sede di Padova.\\
Lo stage riguarda la realizzazione di varie funzionalità per un'applicazione denominata 'Sportwill' che permette la gestione di eventi sportivi.\\
Gli obbiettivi da raggiungere erano molteplici.\\
In primo luogo era richiesto il ripasso del linguaggio Java SE e dei concetti Web come Servlet, servizi Rest e Json.
In secondo luogo era richiesto lo studio dei principi generali, delle best practice, dei widget e dell'architettura di Flutter e lo studio del linguaggio Dart.\\
In seguito si è passati allo studio del codice esistente dell'applicazione e allo sviluppo di varie funzionalità che hanno permesso di completarla rendendola utilizzabile.\\
Il seguente documento è stato diviso in 5 capitoli:
\begin{itemize}
	\item \textbf{Capitolo 1}: Descrizione dell'azienda e delle metodologie utilizzate; \\
	\item \textbf{Capitolo 2}: Presentazione degli obiettivi, del Piano di Lavoro e delle attività svolte con introduzione al progetto; \\
	\item \textbf{Capitolo 3}: Descrizione del linguaggio Dart e del framework Flutter e presentazione di alcune piccole applicazioni realizzate per lo studio; \\
	\item \textbf{Capitolo 4}: Descrizione dettagliata dell'applicazione esistente e delle nuove funzionalità apportate; \\
	\item \textbf{Capitolo 5}: Resoconto conclusivo con valutazione del percorso svolto. \\
\end{itemize}

%\vfill
%
%\selectlanguage{english}
%\pdfbookmark{Abstract}{Abstract}
%\chapter*{Abstract}
%
%\selectlanguage{italian}

\endgroup			

\vfill

